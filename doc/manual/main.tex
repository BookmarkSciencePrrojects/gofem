%% bare_jrnl.tex
%% V1.4b
%% 2015/08/26
%% by Michael Shell
%% see http://www.michaelshell.org/

\documentclass[twoside,journal]{IEEEtran}

\usepackage{mycommands}

\begin{document}



\title{Gofem User Manual}

\author{Dorival~M.~Pedroso%
\thanks{School of Civil Engineering, The University of Queensland, St Lucia QLD 4072, Australia,
e-mail:d.pedroso@uq.edu.au}}

\markboth{User Manual}
{Gofem}

\maketitle




\begin{abstract}

Gofem is a framework to implement solutions of partial differential equations using the finite
element method. Focus is given to solve the equations derived from the theory of porous media.

\end{abstract}

\begin{IEEEkeywords}
partial differential equations, finite element method, parallel computing, porous media mechanics
\end{IEEEkeywords}



\section{Introduction}
\label{sec:intro}

Gofem (Go Finite Element Method) is an implementation of the finite element method (FEM) in Go
language for applications in solid mechanics. The code particularly aims to be as general as
possible and has a focus on porous media mechanics. Nonetheless, classical plasticity and the
solution of multi-physics coupled problems are also targeted by Gofem. Efficiency is also a goal as
long as the quality of code and code maintenance is not penalised. The computational efficiency is
achieved by parallel computing using message passage interface (MPI). Finally, unit tests is
employed for every detail of the code and its use aims to be comprehensive. Gofem depends on the Go
Scientific Library (Gosl) and contributed to the simulations in a number of journal papers,
including \cite{pedroso:15a} and \cite{pedroso:15b}.



\section{Finite element method}

This section presents a quick introduction to the finite element method (FEM) with the equations defined as implemented in Gofem.




\subsection{Pinned frames: Linear elastic rod element}

The element equations are:
\begin{equation}
r_u = K \, u - f
\end{equation}




\subsection{Diffusion equation}



\section{Conclusion}
\label{sec:conclusion}

Gofem is a powerful framework to implement FEM solutions. It is also versatile and easy to be
extended.



\section*{Acknowledgment}

The support from The University of Queensland, Australia and the Australian Research Council are
gratefully acknowledged.



\bibliographystyle{IEEEtran}
\bibliography{mydatabase}

\end{document}
